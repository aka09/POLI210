% Options for packages loaded elsewhere
\PassOptionsToPackage{unicode}{hyperref}
\PassOptionsToPackage{hyphens}{url}
\PassOptionsToPackage{dvipsnames,svgnames*,x11names*}{xcolor}
%
\documentclass[
  ignorenonframetext,
  t]{beamer}
\usepackage{pgfpages}
\setbeamertemplate{caption}[numbered]
\setbeamertemplate{caption label separator}{: }
\setbeamercolor{caption name}{fg=normal text.fg}
\beamertemplatenavigationsymbolsempty
% Prevent slide breaks in the middle of a paragraph
\widowpenalties 1 10000
\raggedbottom
\setbeamertemplate{part page}{
  \centering
  \begin{beamercolorbox}[sep=16pt,center]{part title}
    \usebeamerfont{part title}\insertpart\par
  \end{beamercolorbox}
}
\setbeamertemplate{section page}{
  \centering
  \begin{beamercolorbox}[sep=12pt,center]{part title}
    \usebeamerfont{section title}\insertsection\par
  \end{beamercolorbox}
}
\setbeamertemplate{subsection page}{
  \centering
  \begin{beamercolorbox}[sep=8pt,center]{part title}
    \usebeamerfont{subsection title}\insertsubsection\par
  \end{beamercolorbox}
}
\AtBeginPart{
  \frame{\partpage}
}
\AtBeginSection{
  \ifbibliography
  \else
    \frame{\sectionpage}
  \fi
}
\AtBeginSubsection{
  \frame{\subsectionpage}
}
\usepackage{amsmath,amssymb}
\usepackage{lmodern}
\usepackage{ifxetex,ifluatex}
\ifnum 0\ifxetex 1\fi\ifluatex 1\fi=0 % if pdftex
  \usepackage[T1]{fontenc}
  \usepackage[utf8]{inputenc}
  \usepackage{textcomp} % provide euro and other symbols
\else % if luatex or xetex
  \usepackage{unicode-math}
  \defaultfontfeatures{Scale=MatchLowercase}
  \defaultfontfeatures[\rmfamily]{Ligatures=TeX,Scale=1}
\fi
\usetheme[]{metropolis}
% Use upquote if available, for straight quotes in verbatim environments
\IfFileExists{upquote.sty}{\usepackage{upquote}}{}
\IfFileExists{microtype.sty}{% use microtype if available
  \usepackage[]{microtype}
  \UseMicrotypeSet[protrusion]{basicmath} % disable protrusion for tt fonts
}{}
\makeatletter
\@ifundefined{KOMAClassName}{% if non-KOMA class
  \IfFileExists{parskip.sty}{%
    \usepackage{parskip}
  }{% else
    \setlength{\parindent}{0pt}
    \setlength{\parskip}{6pt plus 2pt minus 1pt}}
}{% if KOMA class
  \KOMAoptions{parskip=half}}
\makeatother
\usepackage{xcolor}
\IfFileExists{xurl.sty}{\usepackage{xurl}}{} % add URL line breaks if available
\IfFileExists{bookmark.sty}{\usepackage{bookmark}}{\usepackage{hyperref}}
\hypersetup{
  pdftitle={POLI210: Political Science Research Methods},
  pdfauthor={Olivier Bergeron-Boutin},
  colorlinks=true,
  linkcolor=Maroon,
  filecolor=Maroon,
  citecolor=Blue,
  urlcolor=blue,
  pdfcreator={LaTeX via pandoc}}
\urlstyle{same} % disable monospaced font for URLs
\newif\ifbibliography
\usepackage{color}
\usepackage{fancyvrb}
\newcommand{\VerbBar}{|}
\newcommand{\VERB}{\Verb[commandchars=\\\{\}]}
\DefineVerbatimEnvironment{Highlighting}{Verbatim}{commandchars=\\\{\}}
% Add ',fontsize=\small' for more characters per line
\usepackage{framed}
\definecolor{shadecolor}{RGB}{48,48,48}
\newenvironment{Shaded}{\begin{snugshade}}{\end{snugshade}}
\newcommand{\AlertTok}[1]{\textcolor[rgb]{1.00,0.81,0.69}{#1}}
\newcommand{\AnnotationTok}[1]{\textcolor[rgb]{0.50,0.62,0.50}{\textbf{#1}}}
\newcommand{\AttributeTok}[1]{\textcolor[rgb]{0.80,0.80,0.80}{#1}}
\newcommand{\BaseNTok}[1]{\textcolor[rgb]{0.86,0.64,0.64}{#1}}
\newcommand{\BuiltInTok}[1]{\textcolor[rgb]{0.80,0.80,0.80}{#1}}
\newcommand{\CharTok}[1]{\textcolor[rgb]{0.86,0.64,0.64}{#1}}
\newcommand{\CommentTok}[1]{\textcolor[rgb]{0.50,0.62,0.50}{#1}}
\newcommand{\CommentVarTok}[1]{\textcolor[rgb]{0.50,0.62,0.50}{\textbf{#1}}}
\newcommand{\ConstantTok}[1]{\textcolor[rgb]{0.86,0.64,0.64}{\textbf{#1}}}
\newcommand{\ControlFlowTok}[1]{\textcolor[rgb]{0.94,0.87,0.69}{#1}}
\newcommand{\DataTypeTok}[1]{\textcolor[rgb]{0.87,0.87,0.75}{#1}}
\newcommand{\DecValTok}[1]{\textcolor[rgb]{0.86,0.86,0.80}{#1}}
\newcommand{\DocumentationTok}[1]{\textcolor[rgb]{0.50,0.62,0.50}{#1}}
\newcommand{\ErrorTok}[1]{\textcolor[rgb]{0.76,0.75,0.62}{#1}}
\newcommand{\ExtensionTok}[1]{\textcolor[rgb]{0.80,0.80,0.80}{#1}}
\newcommand{\FloatTok}[1]{\textcolor[rgb]{0.75,0.75,0.82}{#1}}
\newcommand{\FunctionTok}[1]{\textcolor[rgb]{0.94,0.94,0.56}{#1}}
\newcommand{\ImportTok}[1]{\textcolor[rgb]{0.80,0.80,0.80}{#1}}
\newcommand{\InformationTok}[1]{\textcolor[rgb]{0.50,0.62,0.50}{\textbf{#1}}}
\newcommand{\KeywordTok}[1]{\textcolor[rgb]{0.94,0.87,0.69}{#1}}
\newcommand{\NormalTok}[1]{\textcolor[rgb]{0.80,0.80,0.80}{#1}}
\newcommand{\OperatorTok}[1]{\textcolor[rgb]{0.94,0.94,0.82}{#1}}
\newcommand{\OtherTok}[1]{\textcolor[rgb]{0.94,0.94,0.56}{#1}}
\newcommand{\PreprocessorTok}[1]{\textcolor[rgb]{1.00,0.81,0.69}{\textbf{#1}}}
\newcommand{\RegionMarkerTok}[1]{\textcolor[rgb]{0.80,0.80,0.80}{#1}}
\newcommand{\SpecialCharTok}[1]{\textcolor[rgb]{0.86,0.64,0.64}{#1}}
\newcommand{\SpecialStringTok}[1]{\textcolor[rgb]{0.80,0.58,0.58}{#1}}
\newcommand{\StringTok}[1]{\textcolor[rgb]{0.80,0.58,0.58}{#1}}
\newcommand{\VariableTok}[1]{\textcolor[rgb]{0.80,0.80,0.80}{#1}}
\newcommand{\VerbatimStringTok}[1]{\textcolor[rgb]{0.80,0.58,0.58}{#1}}
\newcommand{\WarningTok}[1]{\textcolor[rgb]{0.50,0.62,0.50}{\textbf{#1}}}
\usepackage{graphicx}
\makeatletter
\def\maxwidth{\ifdim\Gin@nat@width>\linewidth\linewidth\else\Gin@nat@width\fi}
\def\maxheight{\ifdim\Gin@nat@height>\textheight\textheight\else\Gin@nat@height\fi}
\makeatother
% Scale images if necessary, so that they will not overflow the page
% margins by default, and it is still possible to overwrite the defaults
% using explicit options in \includegraphics[width, height, ...]{}
\setkeys{Gin}{width=\maxwidth,height=\maxheight,keepaspectratio}
% Set default figure placement to htbp
\makeatletter
\def\fps@figure{htbp}
\makeatother
\setlength{\emergencystretch}{3em} % prevent overfull lines
\providecommand{\tightlist}{%
  \setlength{\itemsep}{0pt}\setlength{\parskip}{0pt}}
\setcounter{secnumdepth}{-\maxdimen} % remove section numbering
\usepackage{booktabs}
\usepackage{longtable}
\usepackage{array}
\usepackage{multirow}
\usepackage{wrapfig}
\usepackage{float}
\usepackage{colortbl}
\usepackage{pdflscape}
\usepackage{tabu}
\usepackage{threeparttable}
\usepackage{threeparttablex}
\usepackage[normalem]{ulem}
\usepackage{makecell}
\usepackage{xcolor}
\usepackage{fontspec}
\usepackage{animate}
\usepackage{xmpmulti}
\setsansfont[BoldFont={FiraSans-Bold.ttf}]{FiraSans-Light.ttf}
\setmonofont{FiraMono-Regular.ttf}
\usepackage{color}
\definecolor{mygreen}{HTML}{008000}
\ifluatex
  \usepackage{selnolig}  % disable illegal ligatures
\fi
\usepackage[style=authoryear,]{biblatex}
\addbibresource{../210lectures\_bib.bib}

\title{POLI210: Political Science Research Methods}
\subtitle{Lecture 10.2: Measures of association}
\author{Olivier Bergeron-Boutin}
\date{November 4th, 2021}

\begin{document}
\frame{\titlepage}

\begin{frame}[fragile]
\begin{Shaded}
\begin{Highlighting}[]
\CommentTok{\# taken from Andrew Heiss\textquotesingle{} website}
\FunctionTok{library}\NormalTok{(ggtext)}
\NormalTok{theme\_custom }\OtherTok{\textless{}{-}} \ControlFlowTok{function}\NormalTok{()\{}
  \FunctionTok{theme\_minimal}\NormalTok{(}\AttributeTok{base\_size =} \DecValTok{19}\NormalTok{,}
           \AttributeTok{base\_family =} \StringTok{"Fira Sans"}\NormalTok{) }\SpecialCharTok{\%+replace\%}
  \FunctionTok{theme}\NormalTok{(}\AttributeTok{legend.position =} \StringTok{"none"}\NormalTok{,}
        \AttributeTok{panel.grid.minor =} \FunctionTok{element\_blank}\NormalTok{(),}
        \AttributeTok{plot.title =} \FunctionTok{element\_markdown}\NormalTok{(}\AttributeTok{face =} \StringTok{"bold"}\NormalTok{, }\AttributeTok{size =} \FunctionTok{rel}\NormalTok{(}\FloatTok{1.7}\NormalTok{)),}
        \AttributeTok{plot.subtitle =} \FunctionTok{element\_markdown}\NormalTok{(}\AttributeTok{face =} \StringTok{"plain"}\NormalTok{, }\AttributeTok{size =} \FunctionTok{rel}\NormalTok{(}\FloatTok{1.3}\NormalTok{)),}
        \AttributeTok{axis.title =} \FunctionTok{element\_text}\NormalTok{(}\AttributeTok{face =} \StringTok{"bold"}\NormalTok{),}
        \AttributeTok{axis.title.x =} \FunctionTok{element\_text}\NormalTok{(}\AttributeTok{margin =} \FunctionTok{margin}\NormalTok{(}\AttributeTok{t =} \DecValTok{10}\NormalTok{), }\AttributeTok{hjust =} \DecValTok{0}\NormalTok{),}
        \AttributeTok{axis.title.y =} \FunctionTok{element\_text}\NormalTok{(}\AttributeTok{margin =} \FunctionTok{margin}\NormalTok{(}\AttributeTok{r =} \DecValTok{10}\NormalTok{), }\AttributeTok{hjust =} \DecValTok{1}\NormalTok{, }\AttributeTok{angle =} \DecValTok{90}\NormalTok{))}
\NormalTok{\}}
\end{Highlighting}
\end{Shaded}
\end{frame}

\begin{frame}{Some harmless fun}
\protect\hypertarget{some-harmless-fun}{}
\includegraphics[width=\textwidth,height=0.9\textheight]{packages_meme.png}
\end{frame}

\begin{frame}{Boring admin stuff}
\protect\hypertarget{boring-admin-stuff}{}
\begin{itemize}
\tightlist
\item
  Problem set 4 has been posted

  \begin{itemize}
  \tightlist
  \item
    Do it: I take the 3 best grades out of 4 psets; 13.3\% each
  \item
    Don't do it: I take the 3 pset grades; 13.3\% each
  \item
    Due November 15th
  \end{itemize}
\item
  Midterm next week

  \begin{itemize}
  \tightlist
  \item
    A combination of paragraph-length answers and essays
  \item
    Don't lose the forest for the trees!
  \item
    Focus on the broad issues, not on specifics
  \end{itemize}
\end{itemize}
\end{frame}

\begin{frame}{Where we're going}
\protect\hypertarget{where-were-going}{}
We should now be able to describe the distribution of one variable

\begin{itemize}
\tightlist
\item
  The next step: describe how two variables move together
\item
  We will speak of \textbf{correlations}

  \begin{itemize}
  \tightlist
  \item
    When one variable is big/small, does that give me a clue about
    whether some other variable is big/small?
  \end{itemize}
\item
  We want to judge correlations according to two criteria:

  \begin{itemize}
  \tightlist
  \item
    Direction

    \begin{itemize}
    \tightlist
    \item
      Positive correlation: when x is big, y is also big
    \item
      Negative correlation: when x is big, y is small
    \end{itemize}
  \item
    Strength

    \begin{itemize}
    \tightlist
    \item
      How well can I guess the value of y if you give me x?
    \end{itemize}
  \end{itemize}
\item
  The \textbf{correlation coefficient} summarizes both of these

  \begin{itemize}
  \tightlist
  \item
    It's a value between -1 and 1
  \item
    Closer to -1 or 1: stronger relationship

    \begin{itemize}
    \tightlist
    \item
      Correlation of 0: no (linear) relationship
    \end{itemize}
  \item
    The sign indicates the direction
  \end{itemize}
\end{itemize}
\end{frame}

\begin{frame}{Different correlations in scatterplots}
\protect\hypertarget{different-correlations-in-scatterplots}{}
\begin{figure}
\includegraphics[width=1\linewidth]{10.2_Measures_of_association_files/figure-beamer/unnamed-chunk-1-1} \caption{Scatterplots with different correlations}\label{fig:unnamed-chunk-1}
\end{figure}
\end{frame}

\begin{frame}{The scatterplot as a visual tool: economic voting}
\protect\hypertarget{the-scatterplot-as-a-visual-tool-economic-voting}{}
\begin{figure}
\includegraphics[width=0.95\linewidth]{10.2_Measures_of_association_files/figure-beamer/unnamed-chunk-2-1} \caption{Relationship between economic growth and incumbent vote share in the United States, 1792-2016. Data from Guntermann, Lenz, and Myers (2021).}\label{fig:unnamed-chunk-2}
\end{figure}
\end{frame}

\begin{frame}[fragile]{Economic voting}
\protect\hypertarget{economic-voting}{}
The Pearson correlation coefficient:

\scriptsize

\begin{Shaded}
\begin{Highlighting}[]
\FunctionTok{cor}\NormalTok{(economy}\SpecialCharTok{$}\NormalTok{gdpchangeyr3, economy}\SpecialCharTok{$}\NormalTok{partyincshr, }\AttributeTok{use =} \StringTok{"pairwise"}\NormalTok{)}
\end{Highlighting}
\end{Shaded}

\begin{verbatim}
## [1] 0.3763856
\end{verbatim}

\normalsize

A positive, moderately strong relationship

\begin{itemize}
\tightlist
\item
  As GDP growth increases, vote share for the incumbent tends to
  increase as well
\end{itemize}

\begin{table}
\centering
\begin{tabular}[t]{ll}
\toprule
r & Rough meaning\\
\midrule
\cellcolor{gray!6}{+/-0.1-0.3} & \cellcolor{gray!6}{Modest}\\
+/-0.3-0.5 & Moderate\\
\cellcolor{gray!6}{+/-0.5-0.8} & \cellcolor{gray!6}{Strong}\\
+/-0.8-1 & Very strong\\
\bottomrule
\end{tabular}
\end{table}
\end{frame}

\begin{frame}{Economic voting for each party}
\protect\hypertarget{economic-voting-for-each-party}{}
\begin{figure}
\includegraphics[width=0.95\linewidth]{10.2_Measures_of_association_files/figure-beamer/unnamed-chunk-5-1} \caption{Relationship between economic growth and incumbent vote share in the United States, 1792-2016. Data from Guntermann, Lenz, and Myers (2021).}\label{fig:unnamed-chunk-5}
\end{figure}
\end{frame}

\begin{frame}{Economic voting for each party}
\protect\hypertarget{economic-voting-for-each-party-1}{}
\begin{figure}
\includegraphics[width=0.95\linewidth]{10.2_Measures_of_association_files/figure-beamer/unnamed-chunk-6-1} \caption{Relationship between economic growth and incumbent vote share in the United States, 1792-2016. Data from Guntermann, Lenz, and Myers (2021).}\label{fig:unnamed-chunk-6}
\end{figure}
\end{frame}

\begin{frame}[fragile]{Economic voting for each party}
\protect\hypertarget{economic-voting-for-each-party-2}{}
\scriptsize

\begin{Shaded}
\begin{Highlighting}[]
\FunctionTok{library}\NormalTok{(tidyverse)}
\NormalTok{economy }\SpecialCharTok{\%\textgreater{}\%} 
  \FunctionTok{group\_by}\NormalTok{(inc\_party) }\SpecialCharTok{\%\textgreater{}\%} 
  \FunctionTok{summarise}\NormalTok{(}\AttributeTok{cor =} \FunctionTok{cor}\NormalTok{(gdpchangeyr3, partyincshr, }\AttributeTok{use =} \StringTok{"pairwise"}\NormalTok{))}
\end{Highlighting}
\end{Shaded}

\begin{verbatim}
## # A tibble: 3 x 2
##   inc_party    cor
##   <chr>      <dbl>
## 1 Democrat   0.206
## 2 Other      0.432
## 3 Republican 0.593
\end{verbatim}

\normalsize

It looks like the correlation is stronger for Republican incumbents!

Is this a causal relationship? \pause

\begin{itemize}
\tightlist
\item
  Maybe\ldots maybe not!
\item
  We could think of many \textbf{confounders}

  \begin{itemize}
  \tightlist
  \item
    A confounders is related to both X and Y
  \item
    International economy, partisan control of Congress\ldots{}
  \end{itemize}
\end{itemize}
\end{frame}

\begin{frame}{College majors: women and income}
\protect\hypertarget{college-majors-women-and-income}{}
\includegraphics[width=1\linewidth]{10.2_Measures_of_association_files/figure-beamer/unnamed-chunk-8-1}
\end{frame}

\begin{frame}{College majors: women and unemployment}
\protect\hypertarget{college-majors-women-and-unemployment}{}
\includegraphics[width=1\linewidth]{10.2_Measures_of_association_files/figure-beamer/unnamed-chunk-9-1}
\end{frame}

\begin{frame}[fragile]{College majors: correlation coefficients}
\protect\hypertarget{college-majors-correlation-coefficients}{}
\scriptsize

\begin{Shaded}
\begin{Highlighting}[]
\FunctionTok{cor}\NormalTok{(majors}\SpecialCharTok{$}\NormalTok{ShareWomen, majors}\SpecialCharTok{$}\NormalTok{Median, }
    \AttributeTok{use =} \StringTok{"pairwise"}\NormalTok{)}
\end{Highlighting}
\end{Shaded}

\begin{verbatim}
## [1] -0.6186898
\end{verbatim}

\begin{Shaded}
\begin{Highlighting}[]
\FunctionTok{cor}\NormalTok{(majors}\SpecialCharTok{$}\NormalTok{ShareWomen, majors}\SpecialCharTok{$}\NormalTok{Unemployment\_rate, }
    \AttributeTok{use =} \StringTok{"pairwise"}\NormalTok{)}
\end{Highlighting}
\end{Shaded}

\begin{verbatim}
## [1] 0.07320458
\end{verbatim}

\normalsize

\texttt{Share\ of\ women} and \texttt{Median\ salary}: a strong negative
correlation

\texttt{Share\ of\ women} and \texttt{Unemployment}: basically no
association
\end{frame}

\begin{frame}{TV shows}
\protect\hypertarget{tv-shows}{}
\includegraphics[width=1\linewidth]{10.2_Measures_of_association_files/figure-beamer/unnamed-chunk-11-1}
\end{frame}

\begin{frame}[fragile]{TV shows}
\protect\hypertarget{tv-shows-1}{}
\scriptsize

\begin{Shaded}
\begin{Highlighting}[]
\FunctionTok{cor}\NormalTok{(show\_level}\SpecialCharTok{$}\StringTok{\textasciigrave{}}\AttributeTok{1}\StringTok{\textasciigrave{}}\NormalTok{, show\_level}\SpecialCharTok{$}\StringTok{\textasciigrave{}}\AttributeTok{2}\StringTok{\textasciigrave{}}\NormalTok{, }\AttributeTok{use =} \StringTok{"pairwise"}\NormalTok{)}
\end{Highlighting}
\end{Shaded}

\begin{verbatim}
## [1] 0.8274108
\end{verbatim}

\normalsize

Wow, that's a really strong correlation!

\begin{itemize}
\tightlist
\item
  How to interpret?
\item
  Knowing how well-rated the first season is, you can make a very good
  guess as to the rating of the second season
\item
  Do you think the relationship is as strong between season 1 and season
  5?
\end{itemize}
\end{frame}

\begin{frame}{Seasons 1 and 5}
\protect\hypertarget{seasons-1-and-5}{}
\includegraphics[width=1\linewidth]{10.2_Measures_of_association_files/figure-beamer/unnamed-chunk-13-1}
\end{frame}

\begin{frame}[fragile]{Seasons 1 and 5}
\protect\hypertarget{seasons-1-and-5-1}{}
\scriptsize

\begin{Shaded}
\begin{Highlighting}[]
\FunctionTok{cor}\NormalTok{(show\_level\_1\_5}\SpecialCharTok{$}\StringTok{\textasciigrave{}}\AttributeTok{1}\StringTok{\textasciigrave{}}\NormalTok{, show\_level\_1\_5}\SpecialCharTok{$}\StringTok{\textasciigrave{}}\AttributeTok{5}\StringTok{\textasciigrave{}}\NormalTok{, }\AttributeTok{use =} \StringTok{"pairwise"}\NormalTok{)}
\end{Highlighting}
\end{Shaded}

\begin{verbatim}
## [1] 0.6334758
\end{verbatim}

\begin{Shaded}
\begin{Highlighting}[]
\CommentTok{\# you can change the order; doesn\textquotesingle{}t matter}
\FunctionTok{cor}\NormalTok{(show\_level\_1\_5}\SpecialCharTok{$}\StringTok{\textasciigrave{}}\AttributeTok{5}\StringTok{\textasciigrave{}}\NormalTok{, show\_level\_1\_5}\SpecialCharTok{$}\StringTok{\textasciigrave{}}\AttributeTok{1}\StringTok{\textasciigrave{}}\NormalTok{, }\AttributeTok{use =} \StringTok{"pairwise"}\NormalTok{)}
\end{Highlighting}
\end{Shaded}

\begin{verbatim}
## [1] 0.6334758
\end{verbatim}

\normalsize

The correlation is weaker, but still quite strong

\begin{itemize}
\tightlist
\item
  Scatterplots are very useful -- \textbf{always} plot your data
\item
  But must be careful in how you interpret them
\item
  The scale for seasons 1 and 5 is different \(\leadsto\) correlation
  looks weaker than it is
\end{itemize}
\end{frame}

\begin{frame}{Linearity}
\protect\hypertarget{linearity}{}
The correlation coefficient evaluates \textbf{linear} covariation

\begin{itemize}
\tightlist
\item
  What is a linear relationship?
\item
  In response to a change in \(X\), \(Y\) behaves in a particular way,
  \emph{no matter the value of \(X\)}
\item
  Non-linear relationship: the association between \(X\) and \(Y\)
  differs based on the value of \(X\)
\end{itemize}
\end{frame}

\begin{frame}[fragile]{Non-linearity: London Airbnb listings}
\protect\hypertarget{non-linearity-london-airbnb-listings}{}
\begin{figure}
\includegraphics[width=0.85\linewidth]{10.2_Measures_of_association_files/figure-beamer/unnamed-chunk-16-1} \caption{Longitude and price of London (UK) Airbnb listings on March 4th, 2017}\label{fig:unnamed-chunk-16}
\end{figure}

\scriptsize

\begin{Shaded}
\begin{Highlighting}[]
\FunctionTok{cor}\NormalTok{(london}\SpecialCharTok{$}\NormalTok{price, london}\SpecialCharTok{$}\NormalTok{longitude, }\AttributeTok{use =} \StringTok{"pairwise"}\NormalTok{)}
\end{Highlighting}
\end{Shaded}

\begin{verbatim}
## [1] -0.1262614
\end{verbatim}
\end{frame}

\begin{frame}{Equivalent relationships}
\protect\hypertarget{equivalent-relationships}{}
\href{https://media.giphy.com/media/KGYIA3dwufPlN9qNjn/source.gif}{Navigate
to this link}

\begin{itemize}
\tightlist
\item
  For all of these scatterplots, the summary stats are the same!

  \begin{itemize}
  \tightlist
  \item
    Same mean, same correlation, etc.
  \end{itemize}
\item
  And yet, looking at the scatterplots, the relationships are very
  different
\item
  \emph{Always} plot your data!
\item
  Before doing any fancy statistics\ldots{}

  \begin{itemize}
  \tightlist
  \item
    Look at the distribution of \(X\)

    \begin{itemize}
    \tightlist
    \item
      Do any cases stand out?
    \end{itemize}
  \item
    Look at the distribution of \(Y\)

    \begin{itemize}
    \tightlist
    \item
      Do any cases stand out?
    \end{itemize}
  \item
    Look at a scatterplot of \(X\) and \(Y\)

    \begin{itemize}
    \tightlist
    \item
      Do any cases stand out?
    \end{itemize}
  \end{itemize}
\end{itemize}
\end{frame}

\begin{frame}{Not plotting your data? You might screw up}
\protect\hypertarget{not-plotting-your-data-you-might-screw-up}{}
\begin{figure}
\centering
\includegraphics{delao_errors.png}
\caption{See Imai, King, and Velasco Rivera (2020)}
\end{figure}
\end{frame}

\begin{frame}{Scatterplot matrices}
\protect\hypertarget{scatterplot-matrices}{}
\scriptsize

\includegraphics[width=1\linewidth]{10.2_Measures_of_association_files/figure-beamer/unnamed-chunk-18-1}
\end{frame}

\begin{frame}[allowframebreaks]{References}
\protect\hypertarget{references}{}
\footnotesize
\end{frame}

\begin{frame}[allowframebreaks]{}
  \bibliographytrue
  \printbibliography[heading=none]
\end{frame}

\end{document}
