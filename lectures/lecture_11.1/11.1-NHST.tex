% Options for packages loaded elsewhere
\PassOptionsToPackage{unicode}{hyperref}
\PassOptionsToPackage{hyphens}{url}
\PassOptionsToPackage{dvipsnames,svgnames*,x11names*}{xcolor}
%
\documentclass[
  ignorenonframetext,
  t]{beamer}
\usepackage{pgfpages}
\setbeamertemplate{caption}[numbered]
\setbeamertemplate{caption label separator}{: }
\setbeamercolor{caption name}{fg=normal text.fg}
\beamertemplatenavigationsymbolsempty
% Prevent slide breaks in the middle of a paragraph
\widowpenalties 1 10000
\raggedbottom
\setbeamertemplate{part page}{
  \centering
  \begin{beamercolorbox}[sep=16pt,center]{part title}
    \usebeamerfont{part title}\insertpart\par
  \end{beamercolorbox}
}
\setbeamertemplate{section page}{
  \centering
  \begin{beamercolorbox}[sep=12pt,center]{part title}
    \usebeamerfont{section title}\insertsection\par
  \end{beamercolorbox}
}
\setbeamertemplate{subsection page}{
  \centering
  \begin{beamercolorbox}[sep=8pt,center]{part title}
    \usebeamerfont{subsection title}\insertsubsection\par
  \end{beamercolorbox}
}
\AtBeginPart{
  \frame{\partpage}
}
\AtBeginSection{
  \ifbibliography
  \else
    \frame{\sectionpage}
  \fi
}
\AtBeginSubsection{
  \frame{\subsectionpage}
}
\usepackage{amsmath,amssymb}
\usepackage{lmodern}
\usepackage{ifxetex,ifluatex}
\ifnum 0\ifxetex 1\fi\ifluatex 1\fi=0 % if pdftex
  \usepackage[T1]{fontenc}
  \usepackage[utf8]{inputenc}
  \usepackage{textcomp} % provide euro and other symbols
\else % if luatex or xetex
  \usepackage{unicode-math}
  \defaultfontfeatures{Scale=MatchLowercase}
  \defaultfontfeatures[\rmfamily]{Ligatures=TeX,Scale=1}
\fi
\usetheme[]{metropolis}
% Use upquote if available, for straight quotes in verbatim environments
\IfFileExists{upquote.sty}{\usepackage{upquote}}{}
\IfFileExists{microtype.sty}{% use microtype if available
  \usepackage[]{microtype}
  \UseMicrotypeSet[protrusion]{basicmath} % disable protrusion for tt fonts
}{}
\makeatletter
\@ifundefined{KOMAClassName}{% if non-KOMA class
  \IfFileExists{parskip.sty}{%
    \usepackage{parskip}
  }{% else
    \setlength{\parindent}{0pt}
    \setlength{\parskip}{6pt plus 2pt minus 1pt}}
}{% if KOMA class
  \KOMAoptions{parskip=half}}
\makeatother
\usepackage{xcolor}
\IfFileExists{xurl.sty}{\usepackage{xurl}}{} % add URL line breaks if available
\IfFileExists{bookmark.sty}{\usepackage{bookmark}}{\usepackage{hyperref}}
\hypersetup{
  pdftitle={POLI210: Political Science Research Methods},
  pdfauthor={Olivier Bergeron-Boutin},
  colorlinks=true,
  linkcolor=Maroon,
  filecolor=Maroon,
  citecolor=Blue,
  urlcolor=blue,
  pdfcreator={LaTeX via pandoc}}
\urlstyle{same} % disable monospaced font for URLs
\newif\ifbibliography
\usepackage{color}
\usepackage{fancyvrb}
\newcommand{\VerbBar}{|}
\newcommand{\VERB}{\Verb[commandchars=\\\{\}]}
\DefineVerbatimEnvironment{Highlighting}{Verbatim}{commandchars=\\\{\}}
% Add ',fontsize=\small' for more characters per line
\usepackage{framed}
\definecolor{shadecolor}{RGB}{48,48,48}
\newenvironment{Shaded}{\begin{snugshade}}{\end{snugshade}}
\newcommand{\AlertTok}[1]{\textcolor[rgb]{1.00,0.81,0.69}{#1}}
\newcommand{\AnnotationTok}[1]{\textcolor[rgb]{0.50,0.62,0.50}{\textbf{#1}}}
\newcommand{\AttributeTok}[1]{\textcolor[rgb]{0.80,0.80,0.80}{#1}}
\newcommand{\BaseNTok}[1]{\textcolor[rgb]{0.86,0.64,0.64}{#1}}
\newcommand{\BuiltInTok}[1]{\textcolor[rgb]{0.80,0.80,0.80}{#1}}
\newcommand{\CharTok}[1]{\textcolor[rgb]{0.86,0.64,0.64}{#1}}
\newcommand{\CommentTok}[1]{\textcolor[rgb]{0.50,0.62,0.50}{#1}}
\newcommand{\CommentVarTok}[1]{\textcolor[rgb]{0.50,0.62,0.50}{\textbf{#1}}}
\newcommand{\ConstantTok}[1]{\textcolor[rgb]{0.86,0.64,0.64}{\textbf{#1}}}
\newcommand{\ControlFlowTok}[1]{\textcolor[rgb]{0.94,0.87,0.69}{#1}}
\newcommand{\DataTypeTok}[1]{\textcolor[rgb]{0.87,0.87,0.75}{#1}}
\newcommand{\DecValTok}[1]{\textcolor[rgb]{0.86,0.86,0.80}{#1}}
\newcommand{\DocumentationTok}[1]{\textcolor[rgb]{0.50,0.62,0.50}{#1}}
\newcommand{\ErrorTok}[1]{\textcolor[rgb]{0.76,0.75,0.62}{#1}}
\newcommand{\ExtensionTok}[1]{\textcolor[rgb]{0.80,0.80,0.80}{#1}}
\newcommand{\FloatTok}[1]{\textcolor[rgb]{0.75,0.75,0.82}{#1}}
\newcommand{\FunctionTok}[1]{\textcolor[rgb]{0.94,0.94,0.56}{#1}}
\newcommand{\ImportTok}[1]{\textcolor[rgb]{0.80,0.80,0.80}{#1}}
\newcommand{\InformationTok}[1]{\textcolor[rgb]{0.50,0.62,0.50}{\textbf{#1}}}
\newcommand{\KeywordTok}[1]{\textcolor[rgb]{0.94,0.87,0.69}{#1}}
\newcommand{\NormalTok}[1]{\textcolor[rgb]{0.80,0.80,0.80}{#1}}
\newcommand{\OperatorTok}[1]{\textcolor[rgb]{0.94,0.94,0.82}{#1}}
\newcommand{\OtherTok}[1]{\textcolor[rgb]{0.94,0.94,0.56}{#1}}
\newcommand{\PreprocessorTok}[1]{\textcolor[rgb]{1.00,0.81,0.69}{\textbf{#1}}}
\newcommand{\RegionMarkerTok}[1]{\textcolor[rgb]{0.80,0.80,0.80}{#1}}
\newcommand{\SpecialCharTok}[1]{\textcolor[rgb]{0.86,0.64,0.64}{#1}}
\newcommand{\SpecialStringTok}[1]{\textcolor[rgb]{0.80,0.58,0.58}{#1}}
\newcommand{\StringTok}[1]{\textcolor[rgb]{0.80,0.58,0.58}{#1}}
\newcommand{\VariableTok}[1]{\textcolor[rgb]{0.80,0.80,0.80}{#1}}
\newcommand{\VerbatimStringTok}[1]{\textcolor[rgb]{0.80,0.58,0.58}{#1}}
\newcommand{\WarningTok}[1]{\textcolor[rgb]{0.50,0.62,0.50}{\textbf{#1}}}
\usepackage{graphicx}
\makeatletter
\def\maxwidth{\ifdim\Gin@nat@width>\linewidth\linewidth\else\Gin@nat@width\fi}
\def\maxheight{\ifdim\Gin@nat@height>\textheight\textheight\else\Gin@nat@height\fi}
\makeatother
% Scale images if necessary, so that they will not overflow the page
% margins by default, and it is still possible to overwrite the defaults
% using explicit options in \includegraphics[width, height, ...]{}
\setkeys{Gin}{width=\maxwidth,height=\maxheight,keepaspectratio}
% Set default figure placement to htbp
\makeatletter
\def\fps@figure{htbp}
\makeatother
\setlength{\emergencystretch}{3em} % prevent overfull lines
\providecommand{\tightlist}{%
  \setlength{\itemsep}{0pt}\setlength{\parskip}{0pt}}
\setcounter{secnumdepth}{-\maxdimen} % remove section numbering
\usepackage{booktabs}
\usepackage{longtable}
\usepackage{array}
\usepackage{multirow}
\usepackage{wrapfig}
\usepackage{float}
\usepackage{colortbl}
\usepackage{pdflscape}
\usepackage{tabu}
\usepackage{threeparttable}
\usepackage{threeparttablex}
\usepackage[normalem]{ulem}
\usepackage{makecell}
\usepackage{xcolor}
\usepackage{fontspec}
\usepackage{animate}
\usepackage{xmpmulti}
\usepackage{caption}
\setsansfont[BoldFont={FiraSans-Bold.ttf}]{FiraSans-Light.ttf}
\setmonofont{FiraMono-Regular.ttf}
\usepackage{color}
\definecolor{mygreen}{HTML}{008000}
\ifluatex
  \usepackage{selnolig}  % disable illegal ligatures
\fi
\usepackage[style=authoryear,]{biblatex}
\addbibresource{../210lectures\_bib.bib}

\title{POLI210: Political Science Research Methods}
\subtitle{Lecture 11.2: Null Hypothesis Significance Testing}
\author{Olivier Bergeron-Boutin}
\date{November 4th, 2021}

\begin{document}
\frame{\titlepage}

\begin{frame}[fragile]
\begin{Shaded}
\begin{Highlighting}[]
\CommentTok{\# taken from Andrew Heiss\textquotesingle{} website}
\FunctionTok{library}\NormalTok{(ggtext)}
\NormalTok{theme\_custom }\OtherTok{\textless{}{-}} \ControlFlowTok{function}\NormalTok{()\{}
  \FunctionTok{theme\_minimal}\NormalTok{(}\AttributeTok{base\_size =} \DecValTok{19}\NormalTok{,}
           \AttributeTok{base\_family =} \StringTok{"Fira Sans"}\NormalTok{) }\SpecialCharTok{\%+replace\%}
  \FunctionTok{theme}\NormalTok{(}\AttributeTok{legend.position =} \StringTok{"none"}\NormalTok{,}
        \AttributeTok{panel.grid.minor =} \FunctionTok{element\_blank}\NormalTok{(),}
        \AttributeTok{plot.title =} \FunctionTok{element\_markdown}\NormalTok{(}\AttributeTok{face =} \StringTok{"bold"}\NormalTok{, }\AttributeTok{size =} \FunctionTok{rel}\NormalTok{(}\FloatTok{1.7}\NormalTok{)),}
        \AttributeTok{plot.subtitle =} \FunctionTok{element\_markdown}\NormalTok{(}\AttributeTok{face =} \StringTok{"plain"}\NormalTok{, }\AttributeTok{size =} \FunctionTok{rel}\NormalTok{(}\FloatTok{1.3}\NormalTok{)),}
        \AttributeTok{axis.title =} \FunctionTok{element\_text}\NormalTok{(}\AttributeTok{face =} \StringTok{"bold"}\NormalTok{),}
        \AttributeTok{axis.title.x =} \FunctionTok{element\_text}\NormalTok{(}\AttributeTok{margin =} \FunctionTok{margin}\NormalTok{(}\AttributeTok{t =} \DecValTok{10}\NormalTok{), }\AttributeTok{hjust =} \DecValTok{0}\NormalTok{),}
        \AttributeTok{axis.title.y =} \FunctionTok{element\_text}\NormalTok{(}\AttributeTok{margin =} \FunctionTok{margin}\NormalTok{(}\AttributeTok{r =} \DecValTok{10}\NormalTok{), }\AttributeTok{hjust =} \DecValTok{1}\NormalTok{, }\AttributeTok{angle =} \DecValTok{90}\NormalTok{))}
\NormalTok{\}}
\end{Highlighting}
\end{Shaded}
\end{frame}

\begin{frame}{When do you believe me?}
\protect\hypertarget{when-do-you-believe-me}{}
Let's suppose that after the midterm, I tell you that the mean grade is
73

\begin{itemize}
\tightlist
\item
  You suspect that I'm lying, for some reason\ldots{}
\item
  But don't want to call me out unless you're quite sure
\item
  You ask a colleague in lab about their grade\ldots{}

  \begin{itemize}
  \tightlist
  \item
    Then another, and another, and another\ldots{}
  \end{itemize}
\item
  When do you have enough evidence to call me out?
\end{itemize}

\pause
\end{frame}

\begin{frame}{Meeting students}
\protect\hypertarget{meeting-students}{}
\centering

\begin{table}

\caption{\label{tab:unnamed-chunk-1}Grades of students you meet in lab}
\centering
\begin{tabular}[t]{lr}
\toprule
Student \# & Grade\\
\midrule
1 & 63\\
\bottomrule
\end{tabular}
\end{table}

Do you call me a liar?
\end{frame}

\begin{frame}{Meeting students}
\protect\hypertarget{meeting-students-1}{}
\centering

\begin{table}

\caption{\label{tab:unnamed-chunk-2}Grades of students you meet in lab}
\centering
\begin{tabular}[t]{lr}
\toprule
Student \# & Grade\\
\midrule
1 & 63\\
2 & 67\\
\bottomrule
\end{tabular}
\end{table}

Do you call me a liar?
\end{frame}

\begin{frame}{Meeting students}
\protect\hypertarget{meeting-students-2}{}
\centering

\begin{table}

\caption{\label{tab:unnamed-chunk-3}Grades of students you meet in lab}
\centering
\begin{tabular}[t]{lr}
\toprule
Student \# & Grade\\
\midrule
1 & 63\\
2 & 67\\
3 & 71\\
\bottomrule
\end{tabular}
\end{table}

Do you call me a liar?
\end{frame}

\begin{frame}{Meeting students}
\protect\hypertarget{meeting-students-3}{}
\centering

\begin{table}

\caption{\label{tab:unnamed-chunk-4}Grades of students you meet in lab}
\centering
\begin{tabular}[t]{lr}
\toprule
Student \# & Grade\\
\midrule
1 & 63\\
2 & 67\\
3 & 71\\
4 & 56\\
\bottomrule
\end{tabular}
\end{table}

Do you call me a liar?
\end{frame}

\begin{frame}{Meeting students}
\protect\hypertarget{meeting-students-4}{}
\centering

\begin{table}

\caption{\label{tab:unnamed-chunk-5}Grades of students you meet in lab}
\centering
\begin{tabular}[t]{lr}
\toprule
Student \# & Grade\\
\midrule
1 & 63\\
2 & 67\\
3 & 71\\
4 & 56\\
5 & 77\\
\bottomrule
\end{tabular}
\end{table}

Do you call me a liar?
\end{frame}

\begin{frame}{Meeting students}
\protect\hypertarget{meeting-students-5}{}
\centering

\begin{table}

\caption{\label{tab:unnamed-chunk-6}Grades of students you meet in lab}
\centering
\begin{tabular}[t]{lr}
\toprule
Student \# & Grade\\
\midrule
1 & 63\\
2 & 67\\
3 & 71\\
4 & 56\\
5 & 77\\
\addlinespace
6 & 47\\
\bottomrule
\end{tabular}
\end{table}

Do you call me a liar?
\end{frame}

\begin{frame}{The null hypothesis}
\protect\hypertarget{the-null-hypothesis}{}
The setup:

\begin{itemize}
\tightlist
\item
  We set a \textbf{null hypothesis}, also referred to as \(H_0\)

  \begin{itemize}
  \tightlist
  \item
    The null hypothesis is our reference point -- it is arbitrary!
  \item
    It's a sort of statistical ``strawman''
  \end{itemize}
\item
  We then set an \textbf{alternative hypothesis}, or \(H_1\)

  \begin{itemize}
  \tightlist
  \item
    If the null is not true, then the alternative hypothesis must be
    true
  \end{itemize}
\item
  We start from the premise that the null hypothesis is true

  \begin{itemize}
  \tightlist
  \item
    The key question: \emph{How surprised are you to see the data that
    you have, if the null hypothesis is true?}
  \item
    Evidence is inconsistent with the null \(\leadsto\) reject the null
  \item
    Evidence is not inconsistent with the null \(\leadsto\) fail to
    reject the null
  \end{itemize}
\item
  This is the framework of \textbf{hypothesis testing}

  \begin{itemize}
  \tightlist
  \item
    Start from the null
  \item
    Think about what the data should look like, if the null were true
  \item
    Analyze the data; reject/fail to reject the null
  \end{itemize}
\end{itemize}
\end{frame}

\begin{frame}{The null hypothesis in our example}
\protect\hypertarget{the-null-hypothesis-in-our-example}{}
What was the null hypothesis in the example above? \pause

\begin{itemize}
\tightlist
\item
  \(H_0\): \(\mu_{exam} = 73\)
\end{itemize}

What was the alternative hypothesis? \pause

\begin{itemize}
\tightlist
\item
  \(H_1\): \(\mu_{exam} \neq 73\) (non-directional hypothesis)
\item
  \(H_1\): \(\mu_{exam} > 73\) (directional hypothesis)
\item
  \(H_1\): \(\mu_{exam} < 73\) (directional hypothesis) \pause
\end{itemize}
\end{frame}

\begin{frame}{Hypothesis testing in our example}
\protect\hypertarget{hypothesis-testing-in-our-example}{}
Assume that the null is true -- i.e.~the true mean is 73

\begin{itemize}
\tightlist
\item
  What do you expect to see when talking to your peers? \pause

  \begin{itemize}
  \tightlist
  \item
    You expect to see have a sample mean of roughly 73!
  \item
    It might be 71, it might be 75

    \begin{itemize}
    \tightlist
    \item
      Central limit theorem: the sampling distribution is normal and
      centered on the true population parameter
    \end{itemize}
  \item
    But you would be surprised to talk to 20 random students and learn
    that their mean grade is 59
  \item
    The data would be \emph{inconsistent with the null hypothesis}
  \item
    At some point, the data is \emph{so inconsistent with the null
    hypothesis} that we are comfortable rejecting it

    \begin{itemize}
    \tightlist
    \item
      How much we need to see before rejecting the null depends on the
      confidence level that we set
    \end{itemize}
  \end{itemize}
\end{itemize}
\end{frame}

\begin{frame}{Sampling distributions}
\protect\hypertarget{sampling-distributions}{}
If the midterm average really is 73, the sampling distribution should
look like this:

\includegraphics[width=1\linewidth]{11.1-NHST_files/figure-beamer/unnamed-chunk-7-1}
\end{frame}

\begin{frame}{Sampling distributions}
\protect\hypertarget{sampling-distributions-1}{}
I can also show this using a density plot:

\includegraphics[width=0.75\linewidth]{11.1-NHST_files/figure-beamer/unnamed-chunk-8-1}
\end{frame}

\begin{frame}{Sampling distributions with different SD}
\protect\hypertarget{sampling-distributions-with-different-sd}{}
My sampling distribution may have a different standard deviation:

\includegraphics[width=1\linewidth]{11.1-NHST_files/figure-beamer/unnamed-chunk-9-1}
\end{frame}

\begin{frame}{The sampling distribution and hypothesis}
\protect\hypertarget{the-sampling-distribution-and-hypothesis}{}
Whatever the particular SD of sampling distribution\ldots{}

\begin{itemize}
\tightlist
\item
  It should approximate a normal distribution and be centered on the
  true parameter
\item
  The key feature of a normal distribution:

  \begin{itemize}
  \tightlist
  \item
    About 68.4\% of the data is within 1SD of the mean
  \item
    About 95\% of the data is within 2SD of the mean
  \item
    About 99.7\% of the data is within 3SD of the mean
  \end{itemize}
\item
  Therefore, if the null is true, I am\ldots{}

  \begin{itemize}
  \tightlist
  \item
    Not surprised to observe a sample statistic that's 1SD away from the
    null
  \item
    Surprised to observe a sample statistic that's 2 SDs away from the
    null
  \item
    Very surprised to observe a sample statistic that is 3 SDs away from
    the null
  \end{itemize}
\end{itemize}
\end{frame}

\begin{frame}{What does my sampling distribution looks like?}
\protect\hypertarget{what-does-my-sampling-distribution-looks-like}{}
Remember that, in practice, we only draw a single sample

\begin{itemize}
\tightlist
\item
  We \emph{do not} observe the sampling distribution
\item
  But, the sampling distribution has 2 properties:

  \begin{itemize}
  \tightlist
  \item
    The mean

    \begin{itemize}
    \tightlist
    \item
      We will assume that the mean is equal to whatever the null
      hypothesis indicates
    \end{itemize}
  \item
    Standard deviation, for which we have a good guess:

    \begin{itemize}
    \tightlist
    \item
      \(\hat{SE} = \dfrac{\hat{\sigma}}{\sqrt{n}}\)
    \end{itemize}
  \end{itemize}
\item
  With this in mind, we have a good idea of what the sampling
  distribution \emph{should look like} if the null were true

  \begin{itemize}
  \tightlist
  \item
    And therefore we know how unlikely it is to have drawn the sample
    that we drew
  \end{itemize}
\end{itemize}
\end{frame}

\begin{frame}{A hypothetical sampling distribution}
\protect\hypertarget{a-hypothetical-sampling-distribution}{}
\includegraphics[width=1\linewidth]{11.1-NHST_files/figure-beamer/unnamed-chunk-10-1}
\end{frame}

\begin{frame}{A hypothetical sampling distribution}
\protect\hypertarget{a-hypothetical-sampling-distribution-1}{}
\includegraphics[width=1\linewidth]{11.1-NHST_files/figure-beamer/unnamed-chunk-11-1}
\end{frame}

\begin{frame}{A hypothetical sampling distribution}
\protect\hypertarget{a-hypothetical-sampling-distribution-2}{}
\includegraphics[width=1\linewidth]{11.1-NHST_files/figure-beamer/unnamed-chunk-12-1}
\end{frame}

\begin{frame}{But what about small samples?}
\protect\hypertarget{but-what-about-small-samples}{}
Any problem with the previous figure?

\begin{itemize}
\tightlist
\item
  If we draw a single outlying value, should we be surprised?
\item
  Not enough for us to reject the null hypothesis! It's just a single
  value
\item
  So what is the problem with the normal distribution?

  \begin{itemize}
  \tightlist
  \item
    It doesn't take into account sample size
  \end{itemize}
\item
  So instead, we'll use the \textbf{t-distribution}

  \begin{itemize}
  \tightlist
  \item
    It has an additional parameter: \textbf{degrees of freedom}
  \item
    For our purposes, ``degrees of freedom'' refers to sample size
  \item
    With a very high number of degrees of freedom, the t-distribution is
    just like the normal
  \item
    With lower ``df'', the t-distribution has ``fatter tails''
    \(\leadsto\) higher probability of extreme values
  \end{itemize}
\end{itemize}
\end{frame}

\begin{frame}{The t-distribution}
\protect\hypertarget{the-t-distribution}{}
\includegraphics[width=1\linewidth]{11.1-NHST_files/figure-beamer/unnamed-chunk-13-1}
\end{frame}

\begin{frame}{p-values}
\protect\hypertarget{p-values}{}
I now ``know'' what the sampling distribution \emph{would look like}
under the null:

\begin{itemize}
\tightlist
\item
  I know where it peaks (at the null hypothesis, e.g.~\(H_0\):
  \(\mu=73\))
\item
  I ``know'' its standard deviation by estimating the standard error

  \begin{itemize}
  \tightlist
  \item
    \(SE = \dfrac{\hat{\sigma}}{\sqrt{n}}\)
  \end{itemize}
\item
  I know the ``degrees of freedom'' parameter (the sample size)
\end{itemize}

The next step: how likely is the data I observe, if the null is true?

\begin{itemize}
\tightlist
\item
  If \(\mu_{\text{true}} = 73\), I'm not surprised to draw a sample with
  mean - 73
\item
  At some point, the data I observe is so unlikely to have been produced
  by sampling from a population with \(\mu_{\text{true}} = 73\) that I
  must reject the null
\end{itemize}
\end{frame}

\begin{frame}{p-values}
\protect\hypertarget{p-values-1}{}
\begin{itemize}
\tightlist
\item
  Even if I draw a sample that's far from \(H_0\), it's possible I drew
  a weird sample by chance
\item
  e.g.~it is possible to draw 20 random students with
  \(\mu_{\text{grade}} = 62\) even if the true mean is 73
\item
  But there is a point where it's so unlikely that I'm comfortable
  rejecting the null

  \begin{itemize}
  \tightlist
  \item
    This is our prespecified \textbf{significance level} (often
    \(\alpha = 0.05\))
  \end{itemize}
\item
  When looking at our data, we can compute a \textbf{p-value}

  \begin{itemize}
  \tightlist
  \item
    The p-value is a number between 0 and 1
  \item
    It represents the expected probability of observing the sample data,
    if the null hypothesis were true
  \item
    p-value close to 1: given the null, we're not surprised to see this
    \(\leadsto\) fail to reject the null
  \item
    p-value close to 0: given the null, we're surprised to see this
    \(\leadsto\) reject the null
  \item
    \(p < \alpha\): reject the null; \(p > alpha\): fail to reject the
    null
  \end{itemize}
\end{itemize}
\end{frame}

\begin{frame}{Interpreting p-values}
\protect\hypertarget{interpreting-p-values}{}
In our example above, let's say I randomly sample students and compute a
mean grade of 67

\begin{itemize}
\tightlist
\item
  \(H_0\): \(\mu_{\text{true}} = 73\)
\item
  Let's say I get a p-value of 0.13
\item
  What I can say:

  \begin{itemize}
  \tightlist
  \item
    If I were to repeatedly sample from our population (students who
    took the midterm)\ldots{}
  \item
    I would expect to get a result as ``extreme'' as this (extreme = far
    away from the null hypothesis)\ldots{}
  \item
    In about 13\% of repeated samples\ldots{}
  \item
    If the null hypothesis is true
  \end{itemize}
\item
  In other words: it's somewhat unlikely, but very much possible
\item
  With \(\alpha = 0.05\), we fail to reject the null that the mean is 73

  \begin{itemize}
  \tightlist
  \item
    Can we conclude that the mean is 73?

    \begin{itemize}
    \tightlist
    \item
      NO! We do NOT ``accept'' the null; we ``fail to reject''
    \end{itemize}
  \item
    There is no \textbf{statistically significant} difference between
    our sample mean and the null hypothesis
  \end{itemize}
\end{itemize}
\end{frame}

\begin{frame}{p = 0.4}
\protect\hypertarget{p-0.4}{}
\includegraphics[width=1\linewidth]{11.1-NHST_files/figure-beamer/unnamed-chunk-14-1}
\end{frame}

\begin{frame}{p = 0.2}
\protect\hypertarget{p-0.2}{}
\includegraphics[width=1\linewidth]{11.1-NHST_files/figure-beamer/unnamed-chunk-15-1}
\end{frame}

\begin{frame}{p = 0.1}
\protect\hypertarget{p-0.1}{}
\includegraphics[width=1\linewidth]{11.1-NHST_files/figure-beamer/unnamed-chunk-16-1}
\end{frame}

\begin{frame}{p = 0.05}
\protect\hypertarget{p-0.05}{}
\includegraphics[width=1\linewidth]{11.1-NHST_files/figure-beamer/unnamed-chunk-17-1}
\end{frame}

\begin{frame}[fragile]{One-sample t-test in R}
\protect\hypertarget{one-sample-t-test-in-r}{}
\scriptsize

\begin{Shaded}
\begin{Highlighting}[]
\CommentTok{\# the hypothetical grades I gave you earlier}
\NormalTok{grades }\OtherTok{\textless{}{-}} \FunctionTok{c}\NormalTok{(}\DecValTok{63}\NormalTok{, }\DecValTok{67}\NormalTok{, }\DecValTok{71}\NormalTok{, }\DecValTok{56}\NormalTok{, }\DecValTok{77}\NormalTok{, }\DecValTok{47}\NormalTok{)}
\FunctionTok{t.test}\NormalTok{(grades, }\AttributeTok{mu =} \DecValTok{73}\NormalTok{)}
\end{Highlighting}
\end{Shaded}

\begin{verbatim}
## 
##  One Sample t-test
## 
## data:  grades
## t = -2.1615, df = 5, p-value = 0.08303
## alternative hypothesis: true mean is not equal to 73
## 95 percent confidence interval:
##  52.20211 74.79789
## sample estimates:
## mean of x 
##      63.5
\end{verbatim}

\normalsize

Interpret the confidence interval and the p-value

\begin{itemize}
\tightlist
\item
  Should you call me a liar?
\end{itemize}
\end{frame}

\begin{frame}{When should you have called me a liar?}
\protect\hypertarget{when-should-you-have-called-me-a-liar}{}
\begin{table}

\caption{\label{tab:unnamed-chunk-19}Grades of students you meet in lab}
\centering
\begin{tabular}[t]{lrr}
\toprule
Student \# & Grade & p\_value\\
\midrule
1 & 63 & NA\\
2 & 67 & 0.156\\
3 & 71 & 0.122\\
4 & 56 & 0.072\\
5 & 77 & 0.156\\
\addlinespace
6 & 47 & 0.083\\
\textcolor{red}{7} & \textcolor{red}{55} & \textcolor{red}{0.034}\\
\bottomrule
\end{tabular}
\end{table}

You can call me a liar when you get the 7th data point!

\begin{itemize}
\tightlist
\item
  (Assuming \(\alpha = 0.05\))
\end{itemize}
\end{frame}

\begin{frame}{Type I and Type II errors}
\protect\hypertarget{type-i-and-type-ii-errors}{}
When \(p\) is very small, we're very surprised by the data we're seeing

\begin{itemize}
\tightlist
\item
  But weird samples happen!
\item
  It's not \emph{impossible} that the null \emph{is} true given the
  data; it's just very unlikely
\end{itemize}

Take the example above

\begin{itemize}
\tightlist
\item
  If 100 of you talk to peers and ask about their midterm grade
\item
  Each person sets \(\alpha = 0.05\)
\item
  5 people will accuse me of lying \emph{even if the true mean is 73}

  \begin{itemize}
  \tightlist
  \item
    i.e.~they will draw data that is inconsistent with what I said, even
    if what I said is true
  \end{itemize}
\item
  This is \textbf{Type I error}: I reject the null when the null is
  actually true

  \begin{itemize}
  \tightlist
  \item
    Also known as a \textbf{false positive}
  \item
    By setting a lower \(\alpha\), I reduce the chances of Type I errors
  \end{itemize}
\end{itemize}
\end{frame}

\begin{frame}{Type I and Type II errors}
\protect\hypertarget{type-i-and-type-ii-errors-1}{}
\textbf{Type II error} is the opposite

\begin{itemize}
\tightlist
\item
  You fail to reject the null when the null is actually not true
\item
  Also known as a \textbf{false negative}
\item
  By setting a lower \(\alpha\), you \emph{increase} the chances of Type
  II error
\end{itemize}

\centering

\includegraphics[width=\textwidth,height=0.7\textheight]{pregnant_error.png}
\end{frame}

\begin{frame}[fragile]{Differences-in-means}
\protect\hypertarget{differences-in-means}{}
I just presented an example of one hypothesis test where we examined the
mean of a variable against some null hypothesis

\begin{itemize}
\tightlist
\item
  Hypothesis tests can be conducted for many different hypotheses
\item
  Another important application: differences-in-means
\item
  Remember what we did in assignment 2 (causality)?
\end{itemize}

\scriptsize

\begin{Shaded}
\begin{Highlighting}[]
\NormalTok{druckman }\OtherTok{\textless{}{-}} \FunctionTok{read\_csv}\NormalTok{(}\StringTok{"lectures/lecture\_11.1/druckman\_2003.csv"}\NormalTok{)}
\NormalTok{druckman }\SpecialCharTok{\%\textgreater{}\%} 
  \FunctionTok{group\_by}\NormalTok{(tv) }\SpecialCharTok{\%\textgreater{}\%} 
  \FunctionTok{summarise}\NormalTok{(}\AttributeTok{who\_won =} \FunctionTok{mean}\NormalTok{(won2,}\AttributeTok{na.rm =}\NormalTok{ T) }\SpecialCharTok{\%\textgreater{}\%} \FunctionTok{round}\NormalTok{(}\DecValTok{3}\NormalTok{))}
\end{Highlighting}
\end{Shaded}

\begin{verbatim}
## # A tibble: 2 x 2
##      tv who_won
##   <dbl>   <dbl>
## 1     0   0.38 
## 2     1   0.262
\end{verbatim}
\end{frame}

\begin{frame}{The setup}
\protect\hypertarget{the-setup}{}
Again, we have a null hypothesis; what is it? \pause

\begin{itemize}
\tightlist
\item
  \(H_0\): \(\mu_1 = \mu_2\) \pause
\end{itemize}

And we have an alternative hypothesis: \(H_1\): \(\mu_1 \neq \mu_2\)

If the null is true, what do we expect to see?

\begin{itemize}
\tightlist
\item
  If we draw many repeated samples\ldots{}
\item
  And compute the difference-in-means for each\ldots{}
\item
  The sampling distribution should be centered on 0
\end{itemize}

And again, depending on how surprising the data is given the null, we
decide to reject the null or fail to reject it
\end{frame}

\begin{frame}[fragile]{The difference-in-means in R}
\protect\hypertarget{the-difference-in-means-in-r}{}
\scriptsize

\begin{Shaded}
\begin{Highlighting}[]
\FunctionTok{t.test}\NormalTok{(druckman}\SpecialCharTok{$}\NormalTok{won2[druckman}\SpecialCharTok{$}\NormalTok{tv}\SpecialCharTok{==}\DecValTok{0}\NormalTok{], druckman}\SpecialCharTok{$}\NormalTok{won2[druckman}\SpecialCharTok{$}\NormalTok{tv}\SpecialCharTok{==}\DecValTok{1}\NormalTok{])}
\end{Highlighting}
\end{Shaded}

\begin{verbatim}
## 
##  Welch Two Sample t-test
## 
## data:  druckman$won2[druckman$tv == 0] and druckman$won2[druckman$tv == 1]
## t = 3.4387, df = 166.76, p-value = 0.0007382
## alternative hypothesis: true difference in means is not equal to 0
## 95 percent confidence interval:
##  0.05022681 0.18565360
## sample estimates:
## mean of x mean of y 
## 0.3798450 0.2619048
\end{verbatim}
\end{frame}

\begin{frame}[fragile]{The difference-in-means in R}
\protect\hypertarget{the-difference-in-means-in-r-1}{}
\scriptsize

\begin{Shaded}
\begin{Highlighting}[]
\CommentTok{\# equivalent to the above}
\FunctionTok{t.test}\NormalTok{(druckman}\SpecialCharTok{$}\NormalTok{won2 }\SpecialCharTok{\textasciitilde{}}\NormalTok{ druckman}\SpecialCharTok{$}\NormalTok{tv) }\CommentTok{\# mu = 0 is the default}
\end{Highlighting}
\end{Shaded}

\begin{verbatim}
## 
##  Welch Two Sample t-test
## 
## data:  druckman$won2 by druckman$tv
## t = 3.4387, df = 166.76, p-value = 0.0007382
## alternative hypothesis: true difference in means between group 0 and group 1 is not equal to 0
## 95 percent confidence interval:
##  0.05022681 0.18565360
## sample estimates:
## mean in group 0 mean in group 1 
##       0.3798450       0.2619048
\end{verbatim}
\end{frame}

\begin{frame}{Example: difference-in-means}
\protect\hypertarget{example-difference-in-means}{}
\begin{table}

\caption{\label{tab:unnamed-chunk-23}Grades of students you meet in lab}
\centering
\begin{tabular}[t]{lrlr}
\toprule
Male \# & Grade & Female \# & Grade\\
\midrule
1 & 63 & 1 & 68\\
\bottomrule
\end{tabular}
\end{table}
\end{frame}

\begin{frame}[allowframebreaks]{References}
\protect\hypertarget{references}{}
\footnotesize
\end{frame}

\begin{frame}[allowframebreaks]{}
  \bibliographytrue
  \printbibliography[heading=none]
\end{frame}

\end{document}
