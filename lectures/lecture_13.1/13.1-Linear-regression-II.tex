% Options for packages loaded elsewhere
\PassOptionsToPackage{unicode}{hyperref}
\PassOptionsToPackage{hyphens}{url}
\PassOptionsToPackage{dvipsnames,svgnames*,x11names*}{xcolor}
%
\documentclass[
  ignorenonframetext,
  t]{beamer}
\usepackage{pgfpages}
\setbeamertemplate{caption}[numbered]
\setbeamertemplate{caption label separator}{: }
\setbeamercolor{caption name}{fg=normal text.fg}
\beamertemplatenavigationsymbolsempty
% Prevent slide breaks in the middle of a paragraph
\widowpenalties 1 10000
\raggedbottom
\setbeamertemplate{part page}{
  \centering
  \begin{beamercolorbox}[sep=16pt,center]{part title}
    \usebeamerfont{part title}\insertpart\par
  \end{beamercolorbox}
}
\setbeamertemplate{section page}{
  \centering
  \begin{beamercolorbox}[sep=12pt,center]{part title}
    \usebeamerfont{section title}\insertsection\par
  \end{beamercolorbox}
}
\setbeamertemplate{subsection page}{
  \centering
  \begin{beamercolorbox}[sep=8pt,center]{part title}
    \usebeamerfont{subsection title}\insertsubsection\par
  \end{beamercolorbox}
}
\AtBeginPart{
  \frame{\partpage}
}
\AtBeginSection{
  \ifbibliography
  \else
    \frame{\sectionpage}
  \fi
}
\AtBeginSubsection{
  \frame{\subsectionpage}
}
\usepackage{amsmath,amssymb}
\usepackage{lmodern}
\usepackage{ifxetex,ifluatex}
\ifnum 0\ifxetex 1\fi\ifluatex 1\fi=0 % if pdftex
  \usepackage[T1]{fontenc}
  \usepackage[utf8]{inputenc}
  \usepackage{textcomp} % provide euro and other symbols
\else % if luatex or xetex
  \usepackage{unicode-math}
  \defaultfontfeatures{Scale=MatchLowercase}
  \defaultfontfeatures[\rmfamily]{Ligatures=TeX,Scale=1}
\fi
\usetheme[]{metropolis}
% Use upquote if available, for straight quotes in verbatim environments
\IfFileExists{upquote.sty}{\usepackage{upquote}}{}
\IfFileExists{microtype.sty}{% use microtype if available
  \usepackage[]{microtype}
  \UseMicrotypeSet[protrusion]{basicmath} % disable protrusion for tt fonts
}{}
\makeatletter
\@ifundefined{KOMAClassName}{% if non-KOMA class
  \IfFileExists{parskip.sty}{%
    \usepackage{parskip}
  }{% else
    \setlength{\parindent}{0pt}
    \setlength{\parskip}{6pt plus 2pt minus 1pt}}
}{% if KOMA class
  \KOMAoptions{parskip=half}}
\makeatother
\usepackage{xcolor}
\IfFileExists{xurl.sty}{\usepackage{xurl}}{} % add URL line breaks if available
\IfFileExists{bookmark.sty}{\usepackage{bookmark}}{\usepackage{hyperref}}
\hypersetup{
  pdftitle={POLI210: Political Science Research Methods},
  pdfauthor={Olivier Bergeron-Boutin},
  colorlinks=true,
  linkcolor=Maroon,
  filecolor=Maroon,
  citecolor=Blue,
  urlcolor=blue,
  pdfcreator={LaTeX via pandoc}}
\urlstyle{same} % disable monospaced font for URLs
\newif\ifbibliography
\usepackage{color}
\usepackage{fancyvrb}
\newcommand{\VerbBar}{|}
\newcommand{\VERB}{\Verb[commandchars=\\\{\}]}
\DefineVerbatimEnvironment{Highlighting}{Verbatim}{commandchars=\\\{\}}
% Add ',fontsize=\small' for more characters per line
\usepackage{framed}
\definecolor{shadecolor}{RGB}{48,48,48}
\newenvironment{Shaded}{\begin{snugshade}}{\end{snugshade}}
\newcommand{\AlertTok}[1]{\textcolor[rgb]{1.00,0.81,0.69}{#1}}
\newcommand{\AnnotationTok}[1]{\textcolor[rgb]{0.50,0.62,0.50}{\textbf{#1}}}
\newcommand{\AttributeTok}[1]{\textcolor[rgb]{0.80,0.80,0.80}{#1}}
\newcommand{\BaseNTok}[1]{\textcolor[rgb]{0.86,0.64,0.64}{#1}}
\newcommand{\BuiltInTok}[1]{\textcolor[rgb]{0.80,0.80,0.80}{#1}}
\newcommand{\CharTok}[1]{\textcolor[rgb]{0.86,0.64,0.64}{#1}}
\newcommand{\CommentTok}[1]{\textcolor[rgb]{0.50,0.62,0.50}{#1}}
\newcommand{\CommentVarTok}[1]{\textcolor[rgb]{0.50,0.62,0.50}{\textbf{#1}}}
\newcommand{\ConstantTok}[1]{\textcolor[rgb]{0.86,0.64,0.64}{\textbf{#1}}}
\newcommand{\ControlFlowTok}[1]{\textcolor[rgb]{0.94,0.87,0.69}{#1}}
\newcommand{\DataTypeTok}[1]{\textcolor[rgb]{0.87,0.87,0.75}{#1}}
\newcommand{\DecValTok}[1]{\textcolor[rgb]{0.86,0.86,0.80}{#1}}
\newcommand{\DocumentationTok}[1]{\textcolor[rgb]{0.50,0.62,0.50}{#1}}
\newcommand{\ErrorTok}[1]{\textcolor[rgb]{0.76,0.75,0.62}{#1}}
\newcommand{\ExtensionTok}[1]{\textcolor[rgb]{0.80,0.80,0.80}{#1}}
\newcommand{\FloatTok}[1]{\textcolor[rgb]{0.75,0.75,0.82}{#1}}
\newcommand{\FunctionTok}[1]{\textcolor[rgb]{0.94,0.94,0.56}{#1}}
\newcommand{\ImportTok}[1]{\textcolor[rgb]{0.80,0.80,0.80}{#1}}
\newcommand{\InformationTok}[1]{\textcolor[rgb]{0.50,0.62,0.50}{\textbf{#1}}}
\newcommand{\KeywordTok}[1]{\textcolor[rgb]{0.94,0.87,0.69}{#1}}
\newcommand{\NormalTok}[1]{\textcolor[rgb]{0.80,0.80,0.80}{#1}}
\newcommand{\OperatorTok}[1]{\textcolor[rgb]{0.94,0.94,0.82}{#1}}
\newcommand{\OtherTok}[1]{\textcolor[rgb]{0.94,0.94,0.56}{#1}}
\newcommand{\PreprocessorTok}[1]{\textcolor[rgb]{1.00,0.81,0.69}{\textbf{#1}}}
\newcommand{\RegionMarkerTok}[1]{\textcolor[rgb]{0.80,0.80,0.80}{#1}}
\newcommand{\SpecialCharTok}[1]{\textcolor[rgb]{0.86,0.64,0.64}{#1}}
\newcommand{\SpecialStringTok}[1]{\textcolor[rgb]{0.80,0.58,0.58}{#1}}
\newcommand{\StringTok}[1]{\textcolor[rgb]{0.80,0.58,0.58}{#1}}
\newcommand{\VariableTok}[1]{\textcolor[rgb]{0.80,0.80,0.80}{#1}}
\newcommand{\VerbatimStringTok}[1]{\textcolor[rgb]{0.80,0.58,0.58}{#1}}
\newcommand{\WarningTok}[1]{\textcolor[rgb]{0.50,0.62,0.50}{\textbf{#1}}}
\usepackage{graphicx}
\makeatletter
\def\maxwidth{\ifdim\Gin@nat@width>\linewidth\linewidth\else\Gin@nat@width\fi}
\def\maxheight{\ifdim\Gin@nat@height>\textheight\textheight\else\Gin@nat@height\fi}
\makeatother
% Scale images if necessary, so that they will not overflow the page
% margins by default, and it is still possible to overwrite the defaults
% using explicit options in \includegraphics[width, height, ...]{}
\setkeys{Gin}{width=\maxwidth,height=\maxheight,keepaspectratio}
% Set default figure placement to htbp
\makeatletter
\def\fps@figure{htbp}
\makeatother
\setlength{\emergencystretch}{3em} % prevent overfull lines
\providecommand{\tightlist}{%
  \setlength{\itemsep}{0pt}\setlength{\parskip}{0pt}}
\setcounter{secnumdepth}{-\maxdimen} % remove section numbering
\usepackage{booktabs}
\usepackage{longtable}
\usepackage{array}
\usepackage{multirow}
\usepackage{wrapfig}
\usepackage{float}
\usepackage{colortbl}
\usepackage{pdflscape}
\usepackage{tabu}
\usepackage{threeparttable}
\usepackage{threeparttablex}
\usepackage[normalem]{ulem}
\usepackage{makecell}
\usepackage{xcolor}
\usepackage{fontspec}
\usepackage{animate}
\usepackage{xmpmulti}
\usepackage{caption}
\setsansfont[BoldFont={FiraSans-Bold.ttf}]{FiraSans-Light.ttf}
\setmonofont{FiraMono-Regular.ttf}
\usepackage{color}
\definecolor{mygreen}{HTML}{008000}
\ifluatex
  \usepackage{selnolig}  % disable illegal ligatures
\fi
\usepackage[style=authoryear,]{biblatex}
\addbibresource{../210lectures\_bib.bib}

\title{POLI210: Political Science Research Methods}
\subtitle{Lecture 13.2: Linear regression II}
\author{Olivier Bergeron-Boutin}
\date{November 25th, 2021}

\begin{document}
\frame{\titlepage}

\begin{frame}{Boring admin stuff}
\protect\hypertarget{boring-admin-stuff}{}
\begin{itemize}
\tightlist
\item
  Due dates:

  \begin{itemize}
  \tightlist
  \item
    Problem set: December 2nd

    \begin{itemize}
    \tightlist
    \item
      Try to knit right now!
    \end{itemize}
  \item
    Team project: December 6th
  \item
    Quiz: December 2nd to December 6th -- no late penalty until the 17th

    \begin{itemize}
    \tightlist
    \item
      Will be posted soon
    \end{itemize}
  \end{itemize}
\item
  Please complete course evals on Minerva
\end{itemize}
\end{frame}

\begin{frame}{Adding more covariates}
\protect\hypertarget{adding-more-covariates}{}
As seen repeatedly in the class, correlation \(\neq\) causation

\begin{itemize}
\tightlist
\item
  Why? Because of \textbf{confounders}
\item
  But if we can ``adjust'' for all relevant confounders (``control'' for
  them)
\item
  We have a stronger claim to causality
\item
  In addition, from a predictive inference framework, we can make better
  predictions of the value \(Y\) will take on
\end{itemize}

In our regressions, we will include additional covariates

\begin{itemize}
\tightlist
\item
  Covariates = independent variables = explanatory variables
\item
  Just to be clear, we keep the same dependent variable
\item
  But now seek to explain it using multiple variables
\end{itemize}
\end{frame}

\begin{frame}{Our new regression equation}
\protect\hypertarget{our-new-regression-equation}{}
With two independent variables, we now have:

\[Y_i = \beta_0 + \beta_1X_1 + \beta_2X_2 + \epsilon_i\] We still have
our intercept, \(\beta_0\)

\begin{itemize}
\tightlist
\item
  But now we have one coefficient for each independent variable:
  \(\beta_1\) and \(\beta_2\)
\item
  Our interpretation of each coefficient is now a bit different
\item
  \(\beta_1\) representes the expected change in \(Y\) occuring as a
  result of a one-unit change in \(X_1\)\ldots{}\textbf{holding other
  covariates constant}
\item
  In this case, holding \(X_2\) constant
\item
  What we can now say:

  \begin{itemize}
  \tightlist
  \item
    The association between \(X_1\) and \(Y\) that \(\beta_1\)
    identifies is \emph{not} due to confounding by \(X_2\)
  \end{itemize}
\end{itemize}
\end{frame}

\begin{frame}[fragile]{New regression model with incumbent data}
\protect\hypertarget{new-regression-model-with-incumbent-data}{}
\scriptsize

\begin{Shaded}
\begin{Highlighting}[]
\NormalTok{reg2 }\OtherTok{\textless{}{-}} \FunctionTok{lm}\NormalTok{(}\AttributeTok{formula =}\NormalTok{ partyincshr }\SpecialCharTok{\textasciitilde{}}\NormalTok{ gdpchangeyr3 }\SpecialCharTok{+}\NormalTok{ age,}\AttributeTok{data =}\NormalTok{ economy)}
\FunctionTok{summary}\NormalTok{(reg2)}
\end{Highlighting}
\end{Shaded}

\begin{verbatim}
## 
## Call:
## lm(formula = partyincshr ~ gdpchangeyr3 + age, data = economy)
## 
## Residuals:
##      Min       1Q   Median       3Q      Max 
## -13.7772  -3.6559  -0.1206   3.6909  10.6179 
## 
## Coefficients:
##              Estimate Std. Error t value Pr(>|t|)    
## (Intercept)  54.97700    7.56543   7.267 4.65e-09 ***
## gdpchangeyr3  0.63960    0.21801   2.934   0.0053 ** 
## age          -0.08605    0.12965  -0.664   0.5104    
## ---
## Signif. codes:  0 '***' 0.001 '**' 0.01 '*' 0.05 '.' 0.1 ' ' 1
## 
## Residual standard error: 5.581 on 44 degrees of freedom
##   (184 observations deleted due to missingness)
## Multiple R-squared:  0.1664, Adjusted R-squared:  0.1285 
## F-statistic: 4.392 on 2 and 44 DF,  p-value: 0.01823
\end{verbatim}
\end{frame}

\begin{frame}{Comparing our two models}
\protect\hypertarget{comparing-our-two-models}{}
\begin{table}
\centering
\begin{tabular}[t]{lcc}
\toprule
  & Model 1 & Model 2\\
\midrule
(Intercept) & 50.254*** & 54.977***\\
 & (0.999) & (7.565)\\
GDP change (year 3) & 0.605** & 0.640**\\
 & (0.220) & (0.218)\\
Age &  & -0.086\\
 &  & (0.130)\\
\midrule
Num.Obs. & 48 & 47\\
R2 & 0.142 & 0.166\\
R2 Adj. & 0.123 & 0.129\\
\bottomrule
\multicolumn{3}{l}{\rule{0pt}{1em}+ p $<$ 0.1, * p $<$ 0.05, ** p $<$ 0.01, *** p $<$ 0.001}\\
\end{tabular}
\end{table}
\end{frame}

\begin{frame}{Non-linear relationships are not well captured}
\protect\hypertarget{non-linear-relationships-are-not-well-captured}{}
Linear regression models are good at estimating \textbf{linear}
relationships

\begin{itemize}
\tightlist
\item
  When the relationship between \(X\) and \(Y\) is non-linear, things
  get more complicated
\item
  (There are ways to account for this, but that's for 311)
\item
  In short, our \(\beta\)'s will not capture the relationship well
\end{itemize}
\end{frame}

\begin{frame}{Airbnb's in London}
\protect\hypertarget{airbnbs-in-london}{}
\begin{figure}
\includegraphics[width=0.85\linewidth]{13.1-Linear-regression-II_files/figure-beamer/unnamed-chunk-5-1} \caption{Longitude and price of London (UK) Airbnb listings on March 4th, 2017}\label{fig:unnamed-chunk-5}
\end{figure}
\end{frame}

\begin{frame}[fragile]{Airbnb's in London}
\protect\hypertarget{airbnbs-in-london-1}{}
\scriptsize

\begin{verbatim}
## 
## Call:
## lm(formula = price ~ longitude, data = london)
## 
## Residuals:
##     Min      1Q  Median      3Q     Max 
## -117.52  -49.04  -21.41   22.07  893.80 
## 
## Coefficients:
##              Estimate Std. Error t value Pr(>|t|)    
## (Intercept)   79.2799     0.6018  131.75   <2e-16 ***
## longitude   -114.8015     3.8881  -29.53   <2e-16 ***
## ---
## Signif. codes:  0 '***' 0.001 '**' 0.01 '*' 0.05 '.' 0.1 ' ' 1
## 
## Residual standard error: 79.69 on 53815 degrees of freedom
##   (60 observations deleted due to missingness)
## Multiple R-squared:  0.01594,    Adjusted R-squared:  0.01592 
## F-statistic: 871.8 on 1 and 53815 DF,  p-value: < 2.2e-16
\end{verbatim}
\end{frame}

\begin{frame}{Outliers can mess with your results}
\protect\hypertarget{outliers-can-mess-with-your-results}{}
\includegraphics[width=0.95\linewidth]{13.1-Linear-regression-II_files/figure-beamer/unnamed-chunk-7-1}
\end{frame}

\begin{frame}{Outliers can mess with your results}
\protect\hypertarget{outliers-can-mess-with-your-results-1}{}
\includegraphics[width=0.95\linewidth]{13.1-Linear-regression-II_files/figure-beamer/unnamed-chunk-8-1}
\end{frame}

\begin{frame}{Outliers can mess with your results}
\protect\hypertarget{outliers-can-mess-with-your-results-2}{}
\includegraphics[width=0.95\linewidth]{13.1-Linear-regression-II_files/figure-beamer/unnamed-chunk-9-1}
\end{frame}

\begin{frame}{Outliers can mess with your results}
\protect\hypertarget{outliers-can-mess-with-your-results-3}{}
\includegraphics[width=0.95\linewidth]{13.1-Linear-regression-II_files/figure-beamer/unnamed-chunk-10-1}
\end{frame}

\begin{frame}{Outliers can mess with your results}
\protect\hypertarget{outliers-can-mess-with-your-results-4}{}
\begin{table}
\centering
\begin{tabular}[t]{lcc}
\toprule
  & Model 1 & Model 2\\
\midrule
(Intercept) & 38.201*** & 56.773***\\
 & (3.157) & (4.315)\\
GDP change since 1989 & -7.645*** & -23.763***\\
 & (2.178) & (3.415)\\
\midrule
Num.Obs. & 184 & 181\\
R2 & 0.063 & 0.213\\
R2 Adj. & 0.058 & 0.208\\
\bottomrule
\multicolumn{3}{l}{\rule{0pt}{1em}+ p $<$ 0.1, * p $<$ 0.05, ** p $<$ 0.01, *** p $<$ 0.001}\\
\end{tabular}
\end{table}
\end{frame}

\begin{frame}{Outliers are not just a nuisance}
\protect\hypertarget{outliers-are-not-just-a-nuisance}{}
\begin{figure}
\centering
\includegraphics{butterfly_ballot.png}
\caption{The famous butterfly ballot from the 2000 election in Palm
Beach County}
\end{figure}
\end{frame}

\begin{frame}{Outliers are not just a nuisance}
\protect\hypertarget{outliers-are-not-just-a-nuisance-1}{}
\includegraphics[width=0.9\linewidth]{13.1-Linear-regression-II_files/figure-beamer/unnamed-chunk-12-1}
\end{frame}

\begin{frame}{Outliers are not just a nuisance}
\protect\hypertarget{outliers-are-not-just-a-nuisance-2}{}
\includegraphics[width=0.9\linewidth]{13.1-Linear-regression-II_files/figure-beamer/unnamed-chunk-13-1}
\end{frame}

\begin{frame}{Outliers are not just a nuisance}
\protect\hypertarget{outliers-are-not-just-a-nuisance-3}{}
\includegraphics[width=0.9\linewidth]{13.1-Linear-regression-II_files/figure-beamer/unnamed-chunk-14-1}
\end{frame}

\begin{frame}{But generally we should be wary of their influence}
\protect\hypertarget{but-generally-we-should-be-wary-of-their-influence}{}
\centering

\includegraphics{inflation_outlier_1.png}
\end{frame}

\begin{frame}{But generally we should be wary of their influence}
\protect\hypertarget{but-generally-we-should-be-wary-of-their-influence-1}{}
\centering

\includegraphics{inflation_outlier_2.png}
\end{frame}

\begin{frame}{Seems like an important guy\ldots{}}
\protect\hypertarget{seems-like-an-important-guy}{}
\centering

\includegraphics{the_inflation_dude.png}
\end{frame}

\begin{frame}{A note on listwise deletion}
\protect\hypertarget{a-note-on-listwise-deletion}{}
What happens to your regression when the dataset has missing data?

\begin{itemize}
\tightlist
\item
  Listwise deletion: any observation that is missing at least one value
  for any independent variable or the dependent variable will be thrown
  out
\item
  i.e.~the model will not use that observation
\end{itemize}

\begin{table}
\centering
\begin{tabular}[t]{lll}
\toprule
Grade (DV) & Happiness & Hours of sleep\\
\midrule
\cellcolor{gray!6}{87} & \cellcolor{gray!6}{NA} & \cellcolor{gray!6}{7}\\
\midrule
81 & 8 & NA\\
\midrule
\cellcolor{gray!6}{NA} & \cellcolor{gray!6}{6} & \cellcolor{gray!6}{3}\\
\bottomrule
\end{tabular}
\end{table}

Running a model:
\(\text{Sleep}_i = \beta_0 + \beta_1\text{Happiness}_i + \beta_2\text{Sleep}_i + \epsilon_i\)
\end{frame}

\begin{frame}{What's wrong here?}
\protect\hypertarget{whats-wrong-here}{}
\includegraphics{badfigure.png}
\end{frame}

\begin{frame}[fragile]{COVID and democracy}
\protect\hypertarget{covid-and-democracy}{}
\scriptsize

\begin{Shaded}
\begin{Highlighting}[]
\FunctionTok{load}\NormalTok{(}\StringTok{"lectures/lecture\_13.1/survey.RData"}\NormalTok{)}
\CommentTok{\# Model only with anxiety}
\NormalTok{m1\_anxiety }\OtherTok{\textless{}{-}} \FunctionTok{lm}\NormalTok{(}
\NormalTok{  ea\_3item }\SpecialCharTok{\textasciitilde{}}\NormalTok{ anxiety\_scale,}
  \AttributeTok{data =}\NormalTok{ survey}
\NormalTok{)}

\CommentTok{\# formula for the fully{-}specified model}
\NormalTok{reg\_formula }\OtherTok{\textless{}{-}}\NormalTok{ ea\_3item }\SpecialCharTok{\textasciitilde{}}\NormalTok{ anxiety\_scale }\SpecialCharTok{+}\NormalTok{ birth\_decade }\SpecialCharTok{+}\NormalTok{ educ\_4cat }\SpecialCharTok{+}
\NormalTok{  deprivation\_scale }\SpecialCharTok{+}\NormalTok{ authority\_scale\_alt }\SpecialCharTok{+}\NormalTok{ partyid}

\CommentTok{\# Fully{-}specified model}
\NormalTok{m2\_anxiety }\OtherTok{\textless{}{-}} \FunctionTok{lm}\NormalTok{(}
  \AttributeTok{formula =}\NormalTok{ reg\_formula,}
  \AttributeTok{data =}\NormalTok{ survey}
\NormalTok{)}
\end{Highlighting}
\end{Shaded}
\end{frame}

\begin{frame}{Covid and democracy}
\protect\hypertarget{covid-and-democracy-1}{}
\scriptsize

\begin{table}
\centering
\begin{tabular}[t]{lcc}
\toprule
  & Model 1 & Model 2\\
\midrule
COVID-related anxiety & 0.243 (0.014)*** & 0.166 (0.017)***\\
Born in the 1950s &  & -0.017 (0.015)\\
Born in the 1960s &  & 0.004 (0.016)\\
Born in the 1970s &  & 0.050 (0.016)**\\
Born in the 1980s &  & 0.085 (0.016)***\\
Born since 1990 &  & 0.071 (0.015)***\\
Deprivation &  & 0.090 (0.016)***\\
Authoritarianism &  & 0.055 (0.009)***\\
Completed high school &  & -0.010 (0.015)\\
Some postsecondary &  & -0.034 (0.016)*\\
College graduate &  & -0.035 (0.016)*\\
Conservative partisan &  & -0.079 (0.010)***\\
NDP partisan &  & -0.043 (0.013)***\\
Green partisan &  & -0.009 (0.019)\\
Non-partisan &  & -0.035 (0.015)*\\
Constant & 0.255 (0.006)*** & 0.242 (0.019)***\\
\midrule
Num.Obs. & 2417 & 2177\\
R2 & 0.114 & 0.206\\
R2 Adj. & 0.114 & 0.201\\
AIC & -748.6 & -842.6\\
BIC & -731.2 & -746.0\\
Log.Lik. & 377.298 & 438.308\\
F & 310.625 & 37.399\\
Std. Errors & HC0 & HC0\\
\bottomrule
\end{tabular}
\end{table}
\end{frame}

\hypertarget{some-lessons-from-the-class}{%
\section{Some lessons from the
class}\label{some-lessons-from-the-class}}

\begin{frame}{Big takeaways}
\protect\hypertarget{big-takeaways}{}
\begin{itemize}
\tightlist
\item
  Empirical research is hard!

  \begin{itemize}
  \tightlist
  \item
    People who spend their lives doing this get it wrong all the time
  \item
    The first step: recognize how hard this is
  \end{itemize}
\item
  Match the strength of your claims to the strength of your evidence

  \begin{itemize}
  \tightlist
  \item
    Recognize uncertainty
  \item
    When reading about politics in popular media, notice how people
    \emph{don't} do that
  \end{itemize}
\item
  Think about the sort of evidence that would make you change your mind

  \begin{itemize}
  \tightlist
  \item
    If the answer is none\ldots\ldots\ldots\ldots..
  \end{itemize}
\end{itemize}
\end{frame}

\begin{frame}{How can I use this?}
\protect\hypertarget{how-can-i-use-this}{}
To learn more\ldots{}

\begin{itemize}
\tightlist
\item
  POLI311: Quantitative methods
\item
  POLI313: Qualitative methods
\item
  Other than course work: find data that you like!
\end{itemize}

To apply what we've learned\ldots{}

\begin{itemize}
\tightlist
\item
  In popular media:

  \begin{itemize}
  \tightlist
  \item
    How strong are the claims being made
  \item
    How strong is the evidence that is being presented?

    \begin{itemize}
    \tightlist
    \item
      Sometimes, there \emph{is no empirical evidence}; there are entire
      news articles based on the intuition of ``some dude''
    \end{itemize}
  \end{itemize}
\item
  In academics:

  \begin{itemize}
  \tightlist
  \item
    When reading empirical research
  \item
    When reading non-empirical research: what would a good empirical
    test look like?
  \end{itemize}
\end{itemize}
\end{frame}

\begin{frame}{References}
\protect\hypertarget{references}{}
\scriptsize
\end{frame}

\begin{frame}[allowframebreaks]{}
  \bibliographytrue
  \printbibliography[heading=none]
\end{frame}

\end{document}
